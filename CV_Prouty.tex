%%%%%%%%%%%%%%%%%%%%%%%%%%%%%%%
% To Compile to PDF and Open in Background: latex CV.tex && dvipdf CV.dvi && evince CV.pdf &
%%%%%%%%%%%%%%%%%%%%%%%%%%%%%%%
\documentclass[10pt]{article}
\usepackage{array, xcolor, longtable}
\usepackage[margin=1.5cm]{geometry}


\title{\bfseries\huge Roy E. Prouty, Jr.}
\author{}
\date{}

\definecolor{lightgray}{gray}{0.8}

\newcolumntype{L}{>{\raggedleft}p{0.16\textwidth}}

\newcolumntype{R}{p{0.84\textwidth}}

\newcommand\VRule{\color{black}\vrule width 0.5pt}

\begin{document}
\vspace{-1000pt}
\maketitle
\vspace{-57pt}

\begin{center}
\rmfamily {\scshape{1116 South Paca Street \\Baltimore, MD 21230\\
roy.prouty@umbc.edu\\(443) 617-5771\\}}
\vspace{5pt}
%\rmfamily {\textit {Cirriculum Vitae}}
%\rmfamily {\textit {Resume}}
\end{center}
\vspace{-12pt}

\hrule

\section*{Education}
\vspace{-10pt}
\begin {longtable}{L!{\VRule}R}
1/2016 - Present&{\bf University of Maryland, Baltimore County (UMBC)}\\
&{Ph.D. Computer Science (expected 2022)} \\
&{Machine Learning Applications to Astrophysical Datasets}\\
&{Data Assimilation}\\[5pt]
8/2014 - 8/2016&{\bf University of Maryland, Baltimore County}\\
&{M.S. Atmospheric Physics}\\
&{Cloud-Aerosol Microphysics}\\
& {Radiative Transfer}\\[5pt]
9/2009 - 5/2013&{\bf Richard Stockton College of New Jersey (RSC) }(now Stockton University)\\
&{B.S. Applied Physics}\\
&{Minor in Mathematics}\\
&{Wavelet Analysis}\\
&{Analysis of Meteorological Phenomena}\\

\end{longtable}
\vspace{-10pt}

\section*{Professional \& Research}
\vspace{-10pt}
\begin {longtable}{L!{\VRule}R}
8/2017 - Present &{\bf High Performance Research Computing Specialist}\\
& {\bf System Administrator for UMBC HPCF}\\
&{Employed by Division of Information Technology at UMBC (1) deploying and administering HPC clusters, (2) helping students, faculty, and staff accomplish computational research goals on any of the research clusters or machines owned and operated by UMBC.}\\[5pt]

11/2013 - Present&{\bf UMBC Observatory; Director}\\
&{Operating under Center for Space Sciences Technology, lead researcher on DFM $0.8m$,  f/8 research-grade telescope in Physics Building at UMBC. Led various planetary, observational campaigns. Continued work with UMBC Observatory includes weekly public outreach events consisting of Open Houses and/or Public Stargazing.}\\[5pt]

11/2013 - Present&{\bf Computational Spectroscopy \& Planetary Science}\\
&{Work with Dr. Tim Oates, Dr. Susan Hoban, and APL Planetary Astronomer; Stellar Astrophysicist; Infrared Spectroscopist Dr. Carey Lisse on (1) refining stellar classification algorithms and (2) observational campaigns to image comets from UMBC Observatory and investigate stellar atmospheres.}\\[5pt]


6/2017 - 4/2018&{\bf NASA Goddard Spaceflight Center Collaborator \& Intern}\\
&{Continued work with Dr. Jacqueline Le Moigne from NASA GSFC internship addressing development of scalable registration of remote sensing images.}\\[5pt]

1/2016 - 12/2017&{\bf Center for Hybrid Multicore Productivity Research; Research Assistant (RA) }\\
&{Work under Professor Milton Halem, Director of Center for Hybrid Multicore Productivity Research. Experience writing proposals for ROSES and other smaller calls concerning projects that included aiding in development of observation system simulation experiments with NASA Land Information System and development of regression model for CO$_2$ flux inferences using feed-forward neural networks. }\\[5pt]

1/2014 - 8/2016&{\bf Joint Center for Earth Systems Technology RA}\\
%&{Investigation of angular distribution models of radiance fields from above-cloud biomass burning aerosols and development of OMI Absorbing Aerosol Index pipeline for research group. Used Polarized Doubling-Adding Radiative Transfer Model from NASA GISS. Developed elementary Monte Carlo Radiative Transfer Model for multi-level plane-parallel atmospheres.}\\[5pt]
&{Worked with Assistant Professor of Physics Dr. Zhibo Zhang. Investigation of angular distribution models of radiance fields from above-cloud biomass burning aerosols and development of OMI Absorbing Aerosol Index pipeline for research group. Used Polarized Doubling-Adding Radiative Transfer Model from NASA GISS. Developed elementary Monte Carlo Radiative Transfer Model for multi-level plane-parallel atmospheres.}\\[5pt]
\end{longtable}
\vspace{-10pt}

\newpage


\section*{Teaching Experience}
\vspace{-10pt}
\begin{longtable}{L!{\VRule}R}
 10/2016 - Present &{\bf NGSS and K-12 Science Curriculum Consultant}\\
  &{Contracted development of various STEM courses for Maryland Public Schools}\\[5pt]

    1/2014 - Present &{\bf UMBC Guest Lecturer}\\
	&{Guest Lecturer in Physics and Astronomy Courses}\\
	&{Deliver lectures on astronomy in undergraduate courses. Topics ranging from astrobiology, observational astronomy, and galactic astronomy.}\\[5pt]
	 
	   12/2013 - Present &{\bf UMBC Observatory Lecturer}\\
	   &{Delivery of Public Lecture Series on Astrophysics}\\
	   &{Monthly one-hour talks on topics in astrophysics}\\[5pt]

	     1/2018 &{\bf Co-Lead on Educator Professional Development Series}\\
		 &{Delivery of EPD Lectures on Climate Science congruent with Next Generation Science Standards}\\
		 &{Week-long delivery of topics related to climate science via delivery of CHEW(Climate, Health, Ecosystem, Weather) curriculum developed by Dr. Alexandra St. Pé}\\[5pt]

		 5/2016 - 5/2017 &{\bf NASA's BEST Robotics}\\
		 &{Robotics Educator}\\
		 &{Run robotics camps in Maryland County Schools. Develop and deliver structured lectures focused on project-based learning for students between the 8th and 9th grade. The aim of this course is to widen knowledge of basic astronomy, focusing on NASA missions.}\\[5pt]

		  5/2016 - 5/2018&{\bf Education Department at Maryland Academy of Sciences}\\
		   &{Responsible for development and delivery of various astronomy-themed presentations for planetarium display or observatory activities.}\\[5pt]
		    
			 6/2016 - 5/2018 &{\bf Anne Arundel County Public Schools}\\
			   &{Substitute Teacher}\\
			     &{Deliver lessons and minor instruction in mathematics, physics, and computer science.}\\[5pt]
				   
				   &{\bf Teaching Assistantship \dots}\\
				   8/2013 - 12/2014&{{\bf \hspace{10pt}  \dots at UMBC}\newline Worked with Senior Lecturers Eric Anderson, Lili Cui, and Susan Hoban to proctor lab-sections for both algebra and calculus-based introductory physics courses. }\\[5pt]
				   9/2011 - 5/2013&{{\bf \hspace{10pt} \dots at RSC}\newline Oversaw 4-5 physics graders. Developed, proctored, and graded introductory physics exams.}\\[5pt]
				   1/2010 - 5/2013&{\bf Physics Stockroom Technician at RSC}\\
				   &{Responsible for setting up undergraduate physics course laboratories, maintaining equipment, as well as devising, constructing \& carrying-out physics demonstrations for undergraduate physics classes.}\\[5pt]
				   \end{longtable}

				   \section*{Service}
				   \vspace{-10pt}
				   \begin{longtable}{L!{\VRule}R}
				   8/2019 - Present &{\bf UMBC Astronomy Club; Advisor}\\
				   &{Meets regularly with student president and physics department faculty to ensure good communication on Astrophysics Minor. Works with student president to ensure proper procedure for meetings and budgets are followed. Attends and supports Astronomy Club meetings and provides mentorship to undergraduate students.}\\[5pt]
				   7/2018 - 7/2019 &{\bf University System of Maryland (USM) Student Council; President}\\
				   &{Meets regularly with Presidents of all USM Campuses, USM Chancellor, and USM Board of Regents. Responsible for representation and advocacy on behalf of over 176,000 students.  Responsible for familiarity with legislative issues at the state and federal level concerning higher education in Maryland.}\\[5pt]

				   7/2016 - 7/2019 &{\bf Graduate Student Government; President }\\
				   &{Worked with faculty, staff, and students across the university to represent the interests of over 2500 graduate students. Managed and coordinated five executive officers. Responsible for development and execution of \$300,000 annual operating budget. }\\[5pt]

				   9/2017 - 7/2019 &{\bf UMBC Steering Committee; Chair, Committee Member}\\
				   &{Chaired University Steering Committee as coordinating body of Shared Governance at UMBC. Worked with Office of Institutional Advancement and President's Office to support structure of Shared Governance at UMBC.}\\[5pt]
				   \end{longtable}
				   \vspace{-10pt}

				   \newpage

				   \section*{Other}
				   \vspace{-10pt}
				   \begin{longtable}{L!{\VRule}R}
				   8/2018 - Present &{\bf NerdNite Baltimore Boss}\\
				   &{Organized monthly lecture series delivered by Baltimore locals on a variety of topics as lead of local non-profit. Managed logistics, speakers, and funds. }\\[5pt]
				   \end{longtable}


				   \section*{Presentations \& Projects}
				   \vspace{-10pt}
				   \begin{longtable}{L!{\VRule}R}
				   %2017\\&{R. Prouty, W. Forrester, J. Phillips, M. Hymowitz.}{ Development of Honors Astronomy Curriculum for Anne Arundel County Public Schools}.\\[5pt]
				   %2018\\(In preparation)&{R. Prouty, Z. Zhang, H. Yu, and K. Meyer. }{\it Impact of Above Cloud Aerosol on the Angular Distribution Pattern of Cloud Bidirectional Reflectance and Implication for Above Cloud Aerosol Direct Radiative Effect}.\\[5pt]

				   2017\\&{R. Prouty, Jacqueline LeMoigne, Milton Halem.} {\it Efficient Method for Scalable Registration of Remote Sensing Images}. Poster at Fall 2017 AGU Meeting\\[5pt]
				   2016\\&{R. Prouty, Asen Radov. }{\it Inferring CO2 Fluxes from OCO-2 for Assimilation into Land Surface Models to Calculate Net Ecosystem Exchange}. Poster at Fall 2016 AGU Meeting\\[5pt]
				   2016\\&{R. Prouty; M.S. Defense. }{\it Impact of Above Cloud Aerosol on the Angular Distribution Pattern of Cloud Bidirectional Reflectance and Implication for Above Cloud Aerosol Direct Radiative Effect }.\\[5pt]
				   Oct. 2015\\&{N.H. Samarasinha, ..., M. Knight, S. Hoban, R. Prouty et al. }{\it Results from the worldwide coma morphology campaign for comet ISON (C/2012 S1),} Planetary and Space Science.\\[5pt]
				   Jul 28, 2012&{\it The Study of Small Scale Features
				   (Fronts) Found in Long Term Temperature Records}\\
				   &{Poster at American Association of Physics Summer Meeting (Philadelphia, PA); also given at SPS Quadrennial Congress Nov. 9, 2012}\\[5pt]
				   % Nov 9, 2012&{\it Use of Wavelets to Analyze Long Term Temperature Data and Short Term Atmospheric Phenomena}\\
				   % &{Poster at SPS Quadrennial Congress (Orlando, FL)}\\[5pt]
				   Jul - Aug 2012&{\bf Geologic Study Tour of Southwestern United States}\\
				   %Visited southwestern United States with Chinese students from Beijing Normal University led by Professor of Geology Michael Hozik of Richard Stockton College.

				   &{Investigating the geologic structure and history of the Great Salt Lake as well as the canyons of the Colorado Plateau: Bryce, Glen, and Zion; the Grand Canyon.}\\
				   Jul - Aug 2010&{\bf Geologic Study Tour of Northern China}\\
				   &{Visited cities of Datong, Xi'an, and Beijing led by Associate Professor of Physical Geography Weili Qu of Beijing Normal University. Investigating geologic structure and history of northern China.}\\[5pt]

				   \end{longtable}

				   \section*{Conferences \& Workshops}
				           2016, 2017, 2018 American Geophysical Union Fall Meetings. 2016, 2017 NASA Goddard Aersols and Radiation Conference. 2017, 2018. Astronomical Data Analysis Software \& Systems. 2018 Virtual Residency Intermediate Workshop. 2018 SuperComputing. 2019 SuperComputing.

						   \section*{Skills and Interests}

						   \begin{minipage}[ht]{1.0\textwidth}
						   Understanding Complex Systems, Documentation, Python, C, Java, TensorFlow, Fortran, Matlab, GDL/IDL, UNIX OS, Astrophysics, Applied Physics, Atmospheric Physics (Dynamics and Radiative Transfer), Digital Signal Analysis, Neural Networks, Physics Education, Public Speaking, Aviation, German Language \& Culture, \LaTeX.
						   \end{minipage}

						   \end{document}

						   % I would pull out the comet research/observing experience from telescope director and make it its own thing "Astronomical Observations & Research" cuz STScI will care more about that than open house
						   % 
						   % Under programming, list python first "Extensive use of python"   (grammar: "have delivered" not "has" this is YOUR cv -- add IRAF (experience with IRAF) - you need to catch their attention and you can RTFM
						   % 
						   % experience with interfacing instrumentation with Raspberry Pi )ok, it hasn't happened yet, but it will)
						   % 
						   % List your papers, including the one that is in preparation

						   % 
						   % Add your robotics teaching experience w/ AACPS
						   % 
						   % can talk in the evening after the hideous class is over, and gotta go to colloquium...its an Earth science one so I should go
						   % 
						   % falling out now
